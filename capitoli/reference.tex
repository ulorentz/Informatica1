\chapter{La reference online}
In tutti gli esempi affrontati, abbiamo avuto modo di utilizzare funzioni contenute in alcune librerie, dette \emph{standard}, del \verb|C++| (iostream, cstdlib, fstream, \ldots).

La libreria del \verb|C++| è estremamente più estesa di quanto abbiamo visto e usato, ha funzioni di ogni tipo: dai contenitori al multithreading, dai numeri complessi agli algoritmi di base.

Ovviamente, per quanto un programmatore sia esperto, non è richiesta la conoscenza di tutta la libreria: per questo motivo, sono presenti diverse reference online, di cui la più valida forse è \url{http://www.cplusplus.com/reference/}. 

A sinistra, nel menù a tendina, trovi tutte le librerie raggruppate in cinque macro categorie: \emph{C Library} (tutte quelle che iniziano con \verb|<c...>|), \emph{Containers} (classi di vettori, code, liste ecc\ldots), \emph{Input/output} (iostream, fstream, sstream -per lo stream di stringhe-), \emph{multithreading} (libreria piuttosto avanzata che richiederebbe un intero corso  dedicato per poterla usare con disinvoltura) e \emph{other} (contiene un po' di tutto, tra cui cose utilissime ed estremamente interessanti, come \emph{string}, \emph{complex}, \emph{random}, \emph{chrono}, ecc\ldots). 

Prova ad aprire, ad esempio, la macro categoria \emph{C library} e qui selezionare la libreria \emph{ctime}. Ti si aprirà una pagina in cui sono elencate tutte le funzioni, le classi, i tipi di dato e le costanti definite nella libreria. 

Selezioniamo una voce, ad esempio, la funzione \emph{time}. Ti si aprirà una pagina divisa in sezioni: la descrizione generale con il prototipo della funzione; i parametri da passare; il \emph{return value} e, infine, un esempio concreto di come usare la funzione. 

Ti consiglio vivamente di consultare spesso la reference quando vuoi capire di più su funzioni e classi varie o per avere un'idea di cosa offre la libreria del \verb|C++|: programmando tanto è uno strumento fondamentale!

Concludo con un piccolo suggerimento. Spesso ti chiederai ``ma esiste una funzione che faccia\ldots?'', gli informatici ti direbbero: GIYF, ovvero Google Is Your Friend. Cerca su internet e, con ottime probabilità, troverai la risposta alla tua domanda; quindi, una volta scoperto il nome della funzione, puoi controllare sulla reference come si usa. 