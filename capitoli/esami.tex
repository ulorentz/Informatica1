\chapter{Alcuni esami risolti}
\section{Scritti}
\newpage
\section{Laboratorio}

Il concetto è, quindi, dare un'idea di come andrebbe scritto un codice in linea con le richieste di Informatica 1. Ho cercato di usare lo stile e gli strumenti più semplici possibili: il minimo richiesto per superare l'esame a pieni voti, se sai scrivere un codice così e fai tutto giusto il 30 è tuo! 
Volendo si possono usare tanti altri strumenti: classi, template, ecc ecc... Ma se sai queste cose guardare questo svolgimento probabilmente per te è inutile!
A proposito del "tutto giusto": mi sono concentrato sul codice, per quanto riguarda la correttezza dei risultati... non assicuro nulla!


Cosa e' apprezzato nell'esame di Informatica 1, ovvero cosa devi fare per prendere un BEL VOTO?
- I risultati devono essere corretti (ma questo e' ovvio...);
- Il codice deve essere comprensibile e leggibile, potra' sembrarti una cavolata, ma ad ogni parentesi graffa dopo dai i dovuti spazi di tab per renderlo ordinato.
- Il codice deve essere ben strutturato: devi scrivere il meno possibile nel main, dove invece dovresti limitarti a poco piu' di semplici chiamate a funzione. Ogni volta che puoi, e che ha senso farlo, scrivi una funzione che esegua un'insieme di operazioni. 
- Dividi il codice in librerie: le funzioni non tenerle nello stesso file del main, ma crea una, o piu', librerie. Una libreria e' composta da un file ''qualcosa.h'' e un file ''qualcosa.cpp''. Nel .h devono esserci le dichiarazioni delle funzioni, nel .cpp le definizioni. 
- Scrivi un makefile ben fatto e funzionante: compila i file oggetto (compilazione parziale) delle librerie e del main e quindi fai il linking (vedi il file del makefile: li' e' spiegato meglio).
- Stai attento a quelli che possono sembrarti dettagli ma che sono importanti in un buon codice: ogni volta che apri uno stream ad un file controlla che non sia corrotto ma che sia funzionante; quando non ti serve piu' lo stream ricordati di chiuderlo e di non lasciarne nessuno in sospeso; dopo che hai allocato la memoria, quando non e' piu' utile liberala! Non farlo ti penalizzera'! 
- Non fondamentale, ma sicuramente apprezzato: nei passaggi oscuri commenta un minimo il codice per semplificarne la lettura, potrebbe aiutare il prof a capire cosa avevi in mente.
- Non tralasciare ROOT: le prime volte puo' sembrarti arabo, ma e' davvero semplice da usare. Devi solo capire come si usa un istogramma e un plot, di solito sono le uniche due cose che ti vengono richieste. 


Un consiglio: se non sei molto pratico, durante l'esame ogni volta che scrivi un pezzo di codice compilalo e provalo. Non scrivere tutto e alla fine provare a compilare: e' praticamente impossibile scrivere tanto codice senza fare neanche un piccolo errore. Trovare un errore in un pezzetto di codice e' piu' facile che capire perche' un intero programma non funziona. 
Procedi per piccoli passi: scrivi una funzione (tipo carica da file) e testala, se sei sicuro che funziona salva tutto e vai avanti. Se ad un certo punto si rompera' tutto, ci saranno segmentation fault o cose strane, saprai che al passo prima funzionava tutto e sara' facile tornare indietro o aggiustare il nuovo pezzo di codice. 
Non c'e' nulla di peggio di scrivere l'intero programma, testarlo per la prima volta dieci minuti prima della consegna e ritrovarsi come output un segmentation violation.
E, soprattutto, meglio un programma non completo ma che compila di uno che credi essere finito ma che non compila (per Tamascelli, non compila = bocciato)!
\subsection{title}