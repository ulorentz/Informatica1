
\chapter{Strutture di controllo e cicli}\label{cicli}	
Abbiamo visto buona parte di ciò che concerne i dati e la memoria; siamo capaci di usare le variabili e abbiamo capito come vengono trattate nella RAM. Ma, con gli strumenti che abbiamo, il nostro codice non ha molte potenzialità operative, non è particolarmente versatile. 
	
In questo capitolo studieremo due importantissime strutture che rendono il codice potente e dinamico: le strutture di controllo e i cicli. Dal momento che le prime sono spesso utili nelle seconde, inizieremo con queste.
\section{Strutture di controllo}
Ci sono situazioni in cui l'operazione che desideriamo eseguire cambia in base alle condizioni iniziali. Mi spiego meglio: se all'inizio abbiamo una situazione, potremmo voler fare una determinata operazione; se invece la situazione è un'altra, potremmo desiderare una diversa operazione. Come facciamo? Il nostro codice viene eseguito in maniera consequenziale: un'operazione dopo l'altra (una riga di codice dopo l'altra). Come facciamo a dire: ``No, non eseguire quest'operazione, esegui quella!''?\\
	
Per capire meglio ti propongo un esempio pratico: vogliamo scrivere un programma che, preso in input un numero, ne trovi il valore assoluto. 
	
Se viene inserito un numero negativo, ci basta farne il modulo moltiplicando per $-1$, ma se viene inserito un numero positivo? La moltiplicazione per $-1$ non deve assolutamente avvenire!
	
Cerchiamo di estrapolare i concetti generali. Per prima cosa dobbiamo controllare la situazione iniziale, ovvero se si verifica una condizione specifica: nel nostro caso il segno del numero (ad esempio, per condizione prendiamo la negatività del numero). A seconda che la condizione si verifichi o no, dobbiamo eseguire un'istruzione o l'altra.
	
Tornando al nostro esempio: se il numero è negativo (condizione verificata) allora dobbiamo trovarne il valore assoluto, ma se il numero è positivo (condizione non verificata) allora non dobbiamo eseguire nessuna operazione.\\
	
Nota una cosa: il controllo di una condizione è un'operazione binaria. La condizione è verificata o non è verificata, non esiste una terza opzione. Una variabile booleana è perfetta per rappresentare questa situazione: può essere settata su \textbf{true} se la condizione si verifica, \textbf{false} negli altri casi.\\
\subsection{\emph{if\ldots else}} 
Passiamo ad analizzare il codice che rappresenta questa situazione.
\paragraph{if(\ldots)\{\ldots\}}Esiste un costrutto che, a partire da una situazione iniziale, controlla se una condizione viene verificata.  Quindi, se ciò avviene, entra in un nuovo scope ed esegue ciò che è contenuto.
	
Questo costrutto è il seguente:
\begin{lstlisting}
if(condizione della situazione iniziale){//nuovo scope
	//istruzioni
}
\end{lstlisting}
Nota che \textbf{if} è una parola riservata. ``If'' viene seguito da parentesi tonde: al loro interno vi è un codice (generalmente breve) che controlla la situazione iniziale, questo codice deve produrre un risultato tassativamente booleano. Infatti, se nelle parentesi tonde, a controllo concluso, risulta un valore \textbf{true} allora il programma entra nelle parentesi graffe che seguono; se, invece, si trova un valore \textbf{false} ciò non avviene. \\
	
Per capire meglio l'uso di questo enunciato all'interno di un programma, vediamo un esempio pratico: vogliamo scrivere un programma che decida se è pari un numero inserito da tastiera.
	\begin{lstlisting}
#include <iostream>
using namespace std;

int main(){
	int numero=0;
	cout << "Inserisci un numero: " << endl;
	cin >> numero; //cin legge da tastiera un valore e lo inserisce in numero
	if(numero%2==0){
		cout << "Il numero e' pari!" << endl;
	}
	return 0;
}
\end{lstlisting}

Cosa succede?
Fino a  \emph{cin} dovrebbe esserti tutto chiaro. \emph{Cin}, così come \emph{cout}, è un operatore di input/output (o meglio, \emph{cin} di input, \emph{cout} di output). \emph{Cin} è un'abbreviazione di ``console input'' e legge da tastiera un valore, viene seguito da ``>>'' e da una variabile: scrive il valore inserito da tastiera nella variabile. Per una spiegazione più approfondita vedi il capitolo in/out.

A questo punto interviene \textbf{if}: trova il resto della divisione e lo paragona con zero, e se è uguale a zero (il numero è pari) il confronto restituisce \textbf{true}.  Potremmo dire che, in questo caso, \textbf{if} dà l'ok per entrare nelle parentesi graffe: il codice in esse contenuto viene eseguito. Se il resto della divisione non è zero, il confronto con zero dà come risultato \textbf{false}: \textbf{if} non permette di entrare nelle parentesi graffe e il relativo codice non viene eseguito.\\

Perché ho scritto ``=='' e non un semplice ``=''? Questi due simboli, in realtà, sono profondamente diversi. Il primo è un operatore binario\footnote{Con operatore binario, si intende che necessita di due operandi. Un operatore che lavora su un solo operando è detto ``unario''. Ci tengo a sottolineare che ``binario'' non significa, quindi, che opera sui bit. Un operatore che agisce sui bit è detto ``bit a bit'' o, in inglese, \emph{bitwise} (un operatore binario, invece, è detto \emph{binary}).}, ovvero confronta ciò che è a destra con ciò che è a sinistra: se sono uguali restituisce \textbf{true}, altrimenti \textbf{false}. Il semplice ``='', invece, è un operatore di assegnazione: il valore contenuto nella variabile a destra viene assegnato a quella di sinistra. 

Nella prossima sottosezione approfondiremo gli operatori binari e il loro funzionamento. Prima, però, voglio introdurre il completamento naturale della struttura \textbf{if}.

\paragraph{else\{\ldots\}} Nell'esempio precedente controlliamo se una variabile è pari e stampiamo a video questa considerazione. E se volessimo stampare a video qualcosa anche quando la variabile è dispari?
Potremmo pensare di fare una cosa così (ti avviso, è sbagliato!):
\begin{lstlisting}
#include <iostream>
using namespace std;
	
int main(){
	int numero=0;
	cout << "Inserisci un numero: " << endl;
	cin >> numero; //cin legge da tastiera un valore e lo inserisce in numero
	if(numero%2==0){
		cout << "Il numero e' pari!" << endl;
	}
	cout << "Il numero e' dispari!" << endl;
	return 0;
}
	\end{lstlisting}
Se il numero è dispari il programma non entra nello scope di \textbf{if} e tutto va bene: viene stampata la cosa giusta. Ma se il numero è pari? Abbiamo un problema: non solo il programma entra nello scope di \textbf{if}, ma viene stampato anche ciò che segue. Il risultato ottenuto non è quello desiderato!
Potremmo procedere con un secondo \textbf{if} dove, per condizione, andremmo a controllare la cosa \emph{esattamente opposta}. 
	
Quando dopo un \textbf{if} abbiamo un codice che deve essere eseguito al verificarsi della condizione opposta dell'\textbf{if} iniziale, ci viene in aiuto un'altra struttura (in modo da evitare un secondo e ridondante \textbf{if}): \textbf{else}.

Il codice di prima, corretto, diventa:
	\begin{lstlisting}
#include <iostream>
using namespace std;
	
int main(){
	int numero=0;
	cout << "Inserisci un numero: " << endl;
	cin >> numero; //cin legge da tastiera un valore e lo inserisce in numero
	if(numero%2==0){
		cout << "Il numero e' pari!" << endl;
	}
	else{
		cout << "Il numero e' dispari!" << endl;
	}
	return 0;
}
	\end{lstlisting}
	Come funziona \textbf{else}? Si oppone ad un \textbf{if}. Se la condizione di entrata nell'\textbf{if} non è verificata, allora si entra nello scope dell'\textbf{else} (quest'ultimo, quindi, non ha bisogno di alcuna parentesi tonda contenente una condizione). \`E implicito che un \textbf{else} non ha senso se non è preceduto da un \textbf{if}.
	
	\paragraph{NOTA} Una piccola precisazione: se l'istruzione da eseguire dentro un \textbf{if} o un \textbf{else} è costituita da una singola riga di codice (se e solo se è una sola riga!), allora possiamo evitare le parentesi graffe dello scope. Te lo dico perché, per brevità del codice, spesso nei miei esempi scriverò così. Varrà lo stesso per i cicli, ma lo ripeterò. \\
	
	Ti faccio notare che un \textbf{if} non deve per forza essere seguito da un \textbf{else}. Ci sono casi in cui, se la condizione non si verifica, non deve essere eseguito alcun codice alternativo: un caso è l'esempio del valore assoluto che avevo introdotto all'inizio. A dirla, tutta dovresti essere in grado di scrivere il programma da solo (fallo!), in ogni caso ti riporto una possibile soluzione:
	\begin{lstlisting}
#include <iostream>
using namespace std;

int main(){
	int numero=0;
	cout << "Inserisci un numero: " << endl;
	cin >> numero;
	if(numero<0)
		numero=numero*(-1);
	cout << "Valore assoluto: " << numero << endl;
	return 0;		
}
	\end{lstlisting}
	
	Le strutture di controllo possono essere incluse una dentro l'altra, incastrate in una scalata di \textbf{if} ed \textbf{else}; ad esempio:
	\begin{lstlisting}
#include <iostream>
using namespace std;
int main(){
	int numero=0;
	cout << "Inserisci un numero maggiore di zero: " << endl;
	cin >> numero;
	if(numero<10){
		cout << "Il numero ha una singola cifra." << endl;
		if(numero%2==0){
			cout << "Il numero e' pari. " << endl;
			if(numero==2)
				cout << "Il numero e' primo." << endl;
		}
	}
	
	return 0;
}
	\end{lstlisting}
	
	Non è l'unica situazione dove puoi trovare tanti \textbf{if} e, ovviamente, \textbf{else}: nell'esempio di prima per ogni condizione avrei potuto mettere il complementare dentro un \textbf{else}); l'altra situazione è quella, mi vien da dire, ``a cascata'': tanti \textbf{if} scollegati e uno dopo l'altro.
	
	Ti riporto un esempio molto semplice. Vedremo più avanti che esiste una struttura che è in grado di risolvere la stessa situazione in maniera elegante, ma solo in condizioni molto particolari.
	
	\begin{lstlisting}
#include <iostream>
using namespace std;
int main(){
	int c=0;
	cout << "Inserisci un numero da 1 a 5 corrispondente ad una lettera da 'a' a 'e': " << endl;
	cin >> c;
	if(c==1)
		cout << "a!" << endl;
	if(c==2)
		cout << "b!" << endl;
	if(c==3)
		cout << "c!" << endl;
	if(c==4)
		cout << "d!" << endl;
	if(c==5)
		cout << "e!" << endl;
	return 0;
}
	\end{lstlisting}
		
	\subsection{Operatori binari e logici}
	Abbiamo visto che \textbf{if} vuole, nelle parentesi tonde, un enunciato che produca un valore booleano. La cosa più banale è una variabile \textbf{bool}; questa non è, però, l'unica situazione che produce un risultato booleano: molto più notevoli sono gli operatori binari. 
	
	Gli operatori binari valutano una relazione tra due variabili, e a seconda del risultato restituiscono \textbf{true} o \textbf{false}. Nei codici precedenti ne ho introdotti un paio senza molte spiegazioni: vediamoli in dettaglio. 
	
	Chiedersi se una variabile è maggiore (o minore) di un'altra è una relazione binaria: se la variabile è effettivamente maggiore (rispettivamente minore) allora verrà restituito \textbf{true}, al contrario \textbf{false}. 
	
	
	Esistono operatori binari relativi alla matematica ed operatori puramente logici, vediamoli!
	
	Gli operatori binari matematici che andremo ad utilizzare sono i seguenti (il codice in \verb|C++| segue tra virgolette dopo la freccina):
	\begin{itemize}
	\item \textbf{Uguale} $\to$ \textbf{`` == ''}
		\\Come ho già accennato, c'è una profonda differenza tra ``='' e ``==''. Il singolo uguale è un operatore di assegnazione. Ne abbiamo già discusso, non approfondirò di nuovo il significato. I programmatori del \verb|C|, per distinguere l'operatore logico ``uguale'' dall'uguale di assegnazione hanno deciso che, per il primo, il simbolo sarebbe stato ``==''.\\
		Quest'operatore controlla se ciò che è posto a sinistra coincide con ciò che è posto a destra, in tal caso produce un valore \textbf{true}; se ciò non avviene produce un \textbf{false}.\\
		Stai attento a non confondere ``=='' con ``='': se lo fai combini sicuramente un bel disastro nel tuo codice, hai capito perché?\\ \\
		\emph{Importante}: non usare \emph{mai} quest'operatore tra variabili di tipo \textbf{float} o \textbf{double}. Quando abbiamo studiato i dati in virgola mobile, abbiamo visto che non sempre ritroviamo risultati esatti o quello che ci aspettiamo. Un'operazione che dovrebbe generare ``$0$'', nel computer potrebbe produrre un risultato prossimo a tale valore, ma non esattamente zero!!! 
		\\Ti ricordi l'esempio di codice che ti avevo proposto per farti vedere un caso in cui accade esattamente quel problema? Un programma come quello che segue non funzionerà mai:
		\begin{lstlisting}
#include <iostream>
using namespace std;
int main(){
	float a=0.3, b=0.2, c=0.1;
	float d=a-b-c;
	if(d==0)
		cout << "Risultato corretto! " << endl;
	return 0;	
}
		\end{lstlisting}
		Questo è un programma inutile e stupido, ma se nel tuo codice, scrivendo programmi seri, usi dei numeri in virgola mobile con ``=='' potresti creare errori per nulla facili da individuare. \\
		Usa l'operatore ``=='' solo con \textbf{int}, \textbf{short}, \textbf{long}, \textbf{char} e \textbf{bool}.
	 
	\item \textbf{Diverso} $\to$ \textbf{`` != ''}
	\\Non c'è molto da spiegare: è l'esatto opposto dell'operatore precedente. Il punto esclamativo nega ciò che segue: significa ``non uguale''. Questo operatore confronta ciò che è a sinistra con ciò che è a destra: se differiscono produce un valore \textbf{true}.
	\\Giusto per completezza: esiste una parola riservata per indicare ciò che in simboli è ``!='': \textbf{not\_eq}; non ti capiterà praticamente mai di trovarla, ma nel caso sai cos'è\ldots
	\\Vale la stessa considerazione di prima sul non usare numeri in virgola mobile (anche perché, essenzialmente, quest'operatore è identico al precedente: semplicemente produce risultati opposti).
	
	\item \textbf{Maggiore}  $\to$ \textbf{`` > ''}
	\\Funziona esattamente come il relativo operatore matematico. Inoltre, è molto meno problematico nell'uso con i numeri a virgola mobile dei precedenti operatori: l'unico caso critico è quello in cui, di nuovo, due numeri dovrebbero essere uguali (e quindi restituire \textbf{false}) ma in realtà uno è leggermente maggiore dell'altro. \`E comunque un caso raro e, perciò, è comune usare questo operatore anche con i \textbf{float} e i \textbf{double}.
	\item \textbf{Maggiore o uguale}  $\to$ \textbf{`` >= ''}
	\\Anche in questo caso, stesse funzionalità dell'operatore matematico e quasi nessun problema con i \textbf{float} e i \textbf{double}.
	\item \textbf{Minore} $\to$ \textbf{`` < ''}
	\\Stesse considerazioni fatte per ``>''.
	
	\item \textbf{Minore o uguale}  $\to$ \textbf{`` <= ''}
	\\Stesso discorso del ``>=''.
	\end{itemize}
	
	Abbiamo visto praticamente tutti gli operatori binari matematici. Ci sono, però, ancora due operatori binari, diciamo ``logici'': \emph{and} e \emph{or}.
	
	Questi due operatori, a differenza dei precedenti, a sinistra e a destra hanno dei valori booleani e stabiliscono delle relazioni tra gli stessi restituendo un nuovo valore booleano. Vediamo in dettaglio come funzionano.
	\begin{itemize}
	\item \textbf{and}: se sia a destra che a sinistra vi sono due \textbf{true}; restituisce a sua volta \textbf{true}, se i due valori sono diversi o entrambi \textbf{false}, allora restituisce \textbf{false}.
	\\Nel \verb|C| non esisteva la parola riservata \textbf{and}: il simbolo per rappresentarla era ``\&\&''. Nel codice scritto da qualche nostalgico (o pezzi di codice derivanti dal \verb|C|), potresti trovare questo simbolo: non spaventarti, è un semplice \textbf{and}.
	
	
	\item \textbf{or}: affinché sia prodotto un valore \textbf{true}, è sufficiente che solo uno dei due valori sia \textbf{true} (ovviamente possono esserlo entrambi); per cui, l'unico caso in cui viene prodotto un valore \textbf{false} è quando entrambi sono \textbf{false}.
	\\Anche \textbf{or} non esisteva nel \verb|C|: il relativo simbolo era ``||''.
	
	\item Ti aspettavi due punti e invece ecco il terzo!	Voglio fare una precisazione: ``\textbf{!}'' può essere usato in situazioni diverse dal ``!=''. Lo possiamo usare per negare una qualsiasi operazione che produca un valore booleano. 
	\\Il punto esclamativo è, infatti, un operatore logico vero e proprio che nega ciò che segue ma non è un operatore binario: prende un solo argomento.
	\\Di nuovo, per completezza, sappi che esiste una parola riservata per indicare ``!'': \textbf{not}; ti sarà raro trovarla, ma nel caso\ldots
	\\Se scriviamo:
	\begin{lstlisting}
		if(!( a>0 and a<10))
	\end{lstlisting}
	stiamo imponendo che \verb|a| deve essere minore di 0 e maggiore di 10 (estremi inclusi). In questo caso, avremmo potuto semplicemente invertire i versi. Ci sono situazioni, però, in cui torna comodo (ad esempio per negare il valore che ritornano alcune funzioni, nei prossimi capitoli vedrai qualche esempio).\\
	Ovviamente puoi usarlo per negare anche una semplice variabile booleana, esempio:
	\begin{lstlisting}
	bool var=true;
	if(var)
		//codice
	if(!var)
		//codice
	\end{lstlisting}
	Il programma entra nel primo \textbf{if}, mentre nel secondo no.
\end{itemize}
	\textbf{if}, \textbf{and} e \textbf{or} spesso tornano utili per effettuare dei controlli, potremmo dire di ``sicurezza'', sugli input (e varie altre cose). Ad esempio, nel programma scritto prima in cui si chiedeva un numero da 1 a 5 da convertire in lettera, si potrebbe inserire un ``check'' sul fatto che effettivamente il numero inserito sia compreso tra 1 e 5. Provo a farti vedere:
	\begin{lstlisting}
#include <iostream>
using namespace std;
int main(){
	int c=0;
	cout << "Inserisci un numero da 1 a 5 corrispondente ad una lettera da 'a' a 'e': " << endl;
	cin >> c;
	if(c<1 or c>5){
		cout << "Hai inserito un numero non valido!" << endl;
		return 0; //Con questo comando ritorno al sistema operativo 0, interrompo qui il programma, non continua!
	}
	/* L'if precedente si sarebbe potuto scrivere anche cosi':
	if(!(c>=1 and c<=5)){
		cout << "Hai inserito un numero non valido!" << endl;
		return 0;
	}
	Spesso in informatica esistono piu' modi, ugualmente funzionali, per scrivere la stessa cosa: sta al tuo gusto, e a cosa ti viene in mente prima, decidere come comportarti.
	*/
	if(c==1)
		cout << "a!" << endl;
	if(c==2)
		cout << "b!" << endl;
	if(c==3)
		cout << "c!" << endl;
	if(c==4)
		cout << "d!" << endl;
	if(c==5)
		cout << "e!" << endl;
	return 0;
}
	\end{lstlisting}
	
	\section{Cicli}
	Può capitare di dover compiere istruzioni estremamente ripetitive, tutte uguali o nelle quali cambia veramente poco. Invece di scrivere migliaia di righe di codice, è stato introdotto un utilissimo strumento: il ciclo.
	
	Come funziona un ciclo? \`E composto da due parti: un blocco di codice che viene ripetuto in maniera ciclica e un blocco di codice che valuta una condizione: finché la condizione si verifica (o non si verifica, a seconda del tipo di ciclo) il ciclo continua. 
	
	\`E importante notare una cosa: i cicli sono di tre tipi diversi ma sono sempre intercambiabili. Mi spiego meglio: ognuna delle tre tipologie è più adatta ad alcune situazioni, ma ogni ciclo può essere riscritto usandone un altro. 
	
	A volte ti sembrerà che un problema possa essere risolto con cicli diversi, e non stai sbagliando! \`E veramente così (anche se, in alcune situazioni, alcuni cicli sono molto più adatti di altri). La scelta del ciclo è solo tua: da un lato potresti scegliere quello più elegante (che richiede meno righe di codice), dall'altro seguire la tua inclinazione; io, per esempio, tendo ad abusare del ciclo \textbf{for}, probabilmente mi piace più degli altri!
	
	
	\subsection{\emph{while}}
	Per spiegartelo preferisco partire da un esempio di codice, poi lo analizziamo insieme.
	\begin{lstlisting}
#include <iostream>
using namespace std;
int main(){
	int numero=1; //inizializzo ad uno per poter entrare nel ciclo, vedi la spiegazione
	while(numero !=0){
		cout << "Inserisci un numero (zero per chiudere il programma): ";
		cin >> numero;
		if(numero%2==0) //controllo il resto della divisione
			cout << "E' pari!" << endl;
		else
			cout << "E' dispari!" << endl;
	}
	return 0;
}
	\end{lstlisting}
	In breve, il programma funziona così: si avvia, chiede di inserire un numero, analizza se è pari o dispari e continua così finché non si inserisce zero.
	
	Dovrebbe esserti tutto chiaro nel resto del programma, le uniche perplessità potrebbero essere, giustamente, nel \textbf{while}.
	
	Il ciclo \textbf{while} inizia con la relativa parola riservata, quindi è seguito da una parentesi tonda. All'intero di questa vi è un codice che analizza una condizione, e se come risultato viene prodotto un \textbf{true} si entra nelle parentesi graffe: il codice contenuto in esse viene eseguito. Arrivati alla fine delle graffe, \textbf{while} testa di nuovo la condizione, e così via fino a che non si produce un \textbf{false}. In tal caso non si entra più nelle graffe: il programma procede con ciò che segue oltre il ciclo.
	
	Da notare una cosa: prima di entrare nel ciclo si verifica la condizione, e questo vuol dire che potenzialmente non si potrebbe mai entrare nel ciclo. Nell'esempio precedente, se avessi inizializzato \verb|numero| a zero non saremmo mai entrati nel ciclo \textbf{while} e il nostro programma sarebbe stato decisamente inutile.\\
	
	Torniamo all'esempio. In quel caso, qual era il vantaggio e la necessità di usare un ciclo?
	
	Poniamo che il nostro utente sia così negato per la matematica da avere una quantità di numeri non ben definita da processare per decidere se sono pari o dispari (ironizzo: è un modo per dire che l'informatica non è così stupida, si può scrivere di meglio\ldots Ma come al solito cerco esempi banali per scrivere codici brevi). Abbiamo un problema: quando scriviamo il programma non sappiamo quanti numeri vorrà analizzare l'utente, quindi: o lui per ogni numero riavvia il programma (ma sarebbe un programma terribile, nessuno comprerebbe mai qualcosa del genere!) oppure ci ingegniamo per scrivere un programma che si modifichi a run-time, in base alle necessità dell'utente. 
	
	Ecco qui che entra in gioco il ciclo: potenzialmente il programma va avanti all'infinito, continua a chiedere numeri all'utente da analizzare; è quest'ultimo che interrompe il programma quando è soddisfatto.
	
	Esistono molti altri usi di cicli, alcuni dei quali non dipendono dall'utente, ma da condizioni del computer, o da risultati di calcoli effettuati dal computer. A breve cercherò di presentarti qualche esempio più significativo.\\
	
	\paragraph{Scopes e cicli}
	Una caratteristica comune a tutti i cicli è che ogni singola ripetizione rappresenta uno \emph{scope}. Questo cosa significa? Una variabile dichiarata all'interno delle graffe del ciclo è visibile solo e soltanto in quel singolo \emph{scope}: ad ogni ripetizione nasce, viene usata e alla fine della graffa muore; alla ripetizione successiva, nasce di nuovo. Per cui, se avevi assegnato un valore alla ripetizione precedente, non sarà più visibile!
	
	Questo esempio dovrebbe chiarire:
\begin{lstlisting}
#include <iostream>
using namespace std;

int main(){
	int i=0;
	while(i<10){
		int j=1;
		i=j;
		j++; 	
	}
	return 0;
}
\end{lstlisting}
	Il ciclo non si interrompe mai: \verb|j| ad ogni ciclo rinasce e averlo incrementato alla ripetizione precedente non è servito a nulla\footnote{Se un tuo programma entra in \emph{loop} e lo vuoi interrompere, premi Control+C.}. 
	\subsection{\emph{do\ldots while}}
	Nel ciclo \textbf{while}, come abbiamo visto, prima di entrare nelle graffe viene testata la condizione, in questo modo non è detto che si entri nel ciclo. 
	
	Potremmo desiderare, a volte, che avvenga il contrario: per prima cosa si entri, almeno una volta, nel ciclo, dopo di che, per decidere se entrarci di nuovo, dobbiamo testare la condizione, e così via.
	
	Riprendiamo l'esempio di prima: affinché si entrasse nel \textbf{while} abbiamo dovuto inizializzare la variabile \verb|numero| al valore di uno. Dopo un po' che si programma, nel corso di Informatica 1, inizia a diventare automatico inizializzare le variabili numeriche a zero; ricordarsi di inizializzarla ad un valore diverso può essere non del tutto banale. Insomma, tutto questo  per dire: non sarebbe più comodo inizializzare la variabile a zero ed avere un ciclo che, almeno una volta, chiede all'utente di inserire un numero e poi testa la condizione?\\
	
	Esiste il ciclo \textbf{do\ldots while}; riscrivo il programma di prima utilizzandolo, poi, come al solito, spiego tutto.
	
	\begin{lstlisting}
#include <iostream>
using namespace std;
int main(){
	int numero=0;
	do{
		cout << "Inserisci un numero (negativo per chiudere il programma): ";
		cin >> numero;
		if(numero%2==0) //controllo il resto della divisione
			cout << "E' pari!" << endl;
		else
			cout << "E' dispari!" << endl;
	}while(numero >=0);
	return 0;
}
	\end{lstlisting}
	Il programma, arrivato alla parola riservata \textbf{do}, entra nel ciclo, esegue il codice e, solo successivamente, arriva al \textbf{while}, dove testa la condizione e decide se entrare in un nuovo ciclo.
	
	Ti faccio notare una differenza rispetto al ciclo \textbf{while}: nel \textbf{do\ldots while} è \emph{necessario}, alla fine, un bel punto e virgola! \\
	
	Per concludere: i due cicli analizzati finora sono estremamente simili, la differenza è che il secondo esegue almeno una volta il codice tra le graffe.\\
	
	Come accennato, i vari cicli sono interscambiabili. Se vuoi fare un esercizio, per convincerti, prova a scrivere quest'ultima versione del programma (con \verb|numero| inizializzato a 0) con un ciclo \textbf{while}. Dovrai inserire qualcosa di nuovo: prima, dentro o dopo? E cosa?	
	
	\subsection{\emph{for}} 
	Passiamo all'ultimo ciclo, il quale, apparentemente, è tutta un'altra cosa. Un esempio:
\begin{lstlisting}
#include <iostream>
using namespace std;
int main(){
	unsigned int num=0;
	cout << "Programma per conto alla rovescia, da che numero vuoi partire?: ";
	cin >> num;
	for(int i=0; i<num; i++){
		cout << num-i << endl;
	}
	cout << "Zero!" << endl;
	return 0;
}
\end{lstlisting}
	Il ciclo \textbf{for} è composto da due parti: una parentesi tonda e la solita parentesi graffa con il codice da eseguire.
	
	La parentesi tonda, però, è profondamente diversa da quelle degli altri due cicli, vediamo perché.
	
	Come puoi notare, vi sono tre parti distinte separate da un punto e virgola. La prima parte serve per dichiarare ed inizializzare una variabile, detta contatore; generalmente viene chiamata \verb|i|, ma puoi chiamarla come vuoi (se hai tanti \textbf{for} uno dentro l'altro, devi per forza dare nomi diversi); io l'ho inizializzata a zero, ma il valore può essere qualsiasi (spesso si sceglie un valore comodo e sensato per quello che dobbiamo fare). La seconda parte è quella in cui viene testata la condizione, nello specifico che la nostra \verb|i| sia minore di \verb|num|: finché ciò si verifica le iterazioni continuano. La terza parte effettua un'operazione sulla variabile contatore: per esempio io l'ho incrementata (specifico: ad ogni iterazione viene incrementata).
	
	Il concetto è che \verb|i| parte da 0 e ad ogni iterazione si incrementa di 1. Quando arriva al valore \verb|num|-1 (e quindi il programma può ancora entrare nel ciclo) \verb|i| viene incrementato di 1: al successivo controllo non è più minore di \verb|num|, così si esce dal ciclo \textbf{for}.
	
	A volte, le variabili contatore partono da un valore, tipo 10, e vengono decrementate; insomma, non per forza dobbiamo incrementarla! 
	
	Ti accorgerai dell'utilità dei cicli, in particolare del \textbf{for}, parlando di \emph{array}.
	
	Ti faccio notare che, così come con gli \textbf{if}, se abbiamo una sola riga di codice, anche nei cicli possiamo evitare le parentesi graffe.\\
	
	Ora come ora, probabilmente, ti sarà difficile vedere come il ciclo \textbf{for} possa essere equivalente agli altri due. Per capirlo ci serve uno strumento molto utile: il \textbf{break}.
	\subsection{Gli enunciati \emph{break} e \emph{continue}}
	Abbiamo visto che tutti i cicli controllano \emph{una} condizione, solo una! E se noi volessimo controllarne un'altra? Se all'accadere di un particolare evento (uno nuovo, non la condizione del ciclo) volessimo interrompere il ciclo?	\`E possibile: la combinazione di un \textbf{if} e un enunciato \textbf{break} lo permette.
	
	Quando in un ciclo il programma incontra un enunciato \textbf{break}, a qualsiasi iterazione si trovi, il ciclo viene interrotto, si esce dallo \emph{scope} e si continua nel codice del programma.
	
	Un \textbf{break} può essere davvero utilissimo: possiamo usarlo come controllo sulle iterazioni del ciclo (ad esempio interromperlo se dura troppo), oppure per interruzioni forzate da parte dell'utente, ecc\ldots \\
	
	Propongo un esempio: è un programma per calcolare il fattoriale di un numero. Quest'operazione è molto delicata, perché genera numeri enormi con una facilità notevole. Per non rischiare l'overflow possiamo inserire un check nel ciclo: quando il nostro calcolo raggiunge una determinata cifra lo interrompiamo.
	\begin{lstlisting}
#include <iostream>
using namespace std;

int main(){
	unsigned int num=0; //unsigned per avere numero massimo piu' grande!
	unsigned int appo=0; //creo una variabile d'appoggio per il calcolo
	cout << "Inserisci un numero di cui calcolare il fattoriale: ";
	cin >> num;
	appo=1; //per il calcolo del fattoriale
	for(int i=1; i<=num; i++){
		appo=appo*i;
		if(appo>=100000000){ //scelgo questo numero perche' mi sembra ragionevole rispetto alle dimensioni della variabile
			cout << "Numeri troppo grandi!" << endl;
			break;
		}
		if(i==num) //se abbiamo incontrato il break questo non viene eseguito mai
			cout << num << "! = " << appo << endl;
	}
	return 0;
}
	\end{lstlisting}
	Ti accorgerai, se provi a compilare il programma, che con numeri molto piccoli superiamo già il limite imposto.
	
	Potremmo scrivere un controllo sull'overflow decisamente migliore, che ci permetta di utilizzare i limiti dell \textbf{unsigned int} fino in fondo. Se hai voglia, te lo lascio per esercizio!\\ \\
	
	%TODO altro esempio?
	
	Passiamo ora al \textbf{continue}. In realtà è molto più raro da usare, trova molte meno applicazioni, e sinceramente nel corso di Informatica 1 non l'ho mai visto usare. Comunque, io te l'accenno: non si sa mai che tu lo possa trovare utile in qualche tuo codice.
	
	L'enunciato \textbf{continue} dice: interrompi questa iterazione, non continuare con il codice nello scope del ciclo, e vai all'iterazione successiva. In poche parole, a differenza del \textbf{break} non interrompe il ciclo, ma solo la singola iterazione dove viene incontrato: fa passare il programma alla successiva.
	
	Un esempio molto semplice:
	\begin{lstlisting}
#include <iostream>
using namespace std;
int main(){
	int salta=0;
	cout << "Inserisci un numero da 1 a 10: quale numero vuoi saltare nel conto alla rovescia? " ;
	cin >> salta;
	for(int i=10; i>0; i--){
		if(i==salta)
			continue;
		cout << i << endl;
	}
	cout << "Zero! " << endl;
	return 0;
}
	\end{lstlisting}
	
	
	Visti questi elementi, ti è più chiaro come un \textbf{for}, insieme ad un \textbf{if} e \textbf{break}, possa sostituire un \textbf{while}? Fai qualche prova!

	\subsection{I cicli infiniti} 
	Un particolare ciclo è il cosiddetto ``ciclo infinito''. Un ciclo che, in teoria, non si interrompe mai: non esiste una condizione di uscita. O meglio: non esiste come condizione inclusa nel ciclo, ma esiste tramite un \textbf{if}-\textbf{break}. 
	
	I cicli infiniti sono utili quando a priori non sappiamo quanti cicli dovremo fare, oppure se le condizioni di uscita sono tante (e di pari importanza). 
	
	Possiamo scrivere cicli infiniti sia usando \textbf{for} che \textbf{while} (e, ovviamente, \textbf{do\ldots while}):
	\begin{lstlisting}
	//Ciclo for infinito
	for( ; ; ){
		espressione da eseguire;
		if(condizione di uscita)
			break;
		if(altra condizione di uscita)
			break;
		... //Altre condizioni?	
	}
	
	//Ciclo while infinito
	while(true){
		espressione da eseguire;
		if(condizione di uscita)
			break;
		if(altra condizione di uscita)
			break;
		... //Altre condizioni?		
	}
	\end{lstlisting}
	
	Ho messo più condizioni per rendere chiaro che possono essere tante, in realtà ne basta una sola (ma almeno una\ldots se no, buona attesa per la fine del programma!).
	
	Il ciclo \textbf{for} infinito ha un vantaggio: può implementare in maniera elegante un contatore di iterazioni. Ti mostro:
	\begin{lstlisting}
	for(int i=0; ; i++){
		espressione;
		if(condizione di uscita)
			break;	
	}
	\end{lstlisting}
	
	Mancando la parte centrale del \textbf{for}, non vi è un limite alle iterazioni; la presenza della variabile \verb|i|, invece, serve da contatore: può essere utile all'interno del ciclo.
	
	Ovviamente, in tutti questi cicli, l'\textbf{if} non deve essere per forza alla fine, ma lo puoi posizionare all'inizio, in mezzo, dove vuoi!\\
	
	Passiamo al dovuto esempio di codice:
	\begin{lstlisting}
#include <iostream>
using namespace std;
int main(){
	const int numero=17;
	int risposta=0;
	cout << "Indovina il numero segreto! Prova con un numero da 1 a 20. Inserisci 0 per chiudere il programma." << endl;
	for( ; ; ){
		cout << "Inserisci un tentativo: ";
		cin >> risposta;
		if(risposta == 0)
			return 0; //chiudo direttamente il programma
		if(risposta == numero)
			break; //interrompo il ciclo infinito
		cout << "Hai sbagliato, prova di nuovo! " << endl;
	}
	cout << "Bravo, hai indovinato! " << endl;
	return 0;
}
	\end{lstlisting}
	Come vedi, posso uscire da un ciclo non solo con l'enunciato \textbf{break}, ma anche con un \textbf{return} (che, in questo caso, chiude il programma).
	
	L'utilità del ciclo infinito, in questo caso, è che abbiamo due condizioni di uscita di pari importanza. \`E, in realtà, una pura questione di stile: avremmo potuto scrivere un normalissimo \textbf{while} e un ulteriore \textbf{if} per la seconda condizione di uscita. Sta sempre a te decidere come comportarti (anche se ci sono occasioni in cui il ciclo infinito è davvero più comodo)!
	
	Dovrebbe essere tutto chiaro, così come il fatto che possiamo complicare il programma inserendo un \textbf{if} per controllare che l'input sia effettivamente un numero tra 0 e 20 ed avvertire l'utente in caso contrario.
	
	\subsection{\emph{switch}\ldots\emph{case}}
	Questo costrutto schematizza una serie di \textbf{if}. Si presta per risolvere in maniera elegante alcune situazioni particolari. 
	
	Ad esempio, se abbiamo una variabile che può assumere valori discreti e per ogni valore vogliamo accada una cosa diversa, lo \textbf{switch-case} è perfetto! \`E un costrutto che si presta benissimo per situazione di ``multi selezione''. 
	
	Ma come è strutturato? In generale è composto così:
	\begin{lstlisting}
	switch(variabile){
		case valore-intero-1:
			espressione;
			break;
		case valore-intero-2:
			espressione;
			break;
		...
		...
		...
		case valore-intero-n:
			espressione;
			break;
		default:
			espressione;	
	}
	\end{lstlisting}
	
	Subito dopo \textbf{switch} vi è una parentesi tonda, all'interno della quale c'è la variabile che può assumere una serie di valori discreti. Segue una parentesi graffa, in cui abbiamo diversi \textbf{case}, uno per ogni valore. Dopo ogni \textbf{case}, infatti, vi è il valore, che può essere o un numero intero o una lettera (la quale va racchiusa tra singoli apici, come `a'). Nota che, quello che segue a \textbf{case}, deve essere un valore costante: non può essere un'espressione del tipo "\verb|nuovavariabile|/17" (che dipende dal valore, variabile, di \verb|nuovavariabile|). Il valore viene concluso da due punti, dopo i quali vi è l'espressione che vogliamo venga eseguita. Alla fine vi è un \textbf{break}: la sua presenza ci porta alla fine della parentesi graffa.  Senza di lui passeremmo al \textbf{case} successivo, anche se non è quello desiderato.
	
	Particolare è la presenza del \textbf{default}: se la variabile assume un valore diverso da quello di ogni \textbf{case}, viene eseguito il codice che segue al \textbf{default} (può anche non seguire niente, se non vogliamo venga eseguito qualcosa).
	
	Questa struttura è particolarmente comoda per menù o situazioni di scelta multipla.
	
	Ti ripropongo il programma per stampare a video le lettere che avevo presentato con l'\textbf{if}. L'ho scritto utilizzando molti elementi presentati finora: ciclo \textbf{for} infinito, \textbf{break} e \textbf{continue}. Dai un occhio: spero ti sia tutto chiaro!
	\begin{lstlisting}
#include <iostream>
using namespace std;
int main(){
	int c=0;
	for( ; ; ){
		cout << "Inserisci un numero da 1 a 5 corrispondente ad una lettera da 'a' a 'e': ";
		cin >> c;
		switch(c){
			case 1:
				cout << "a!" << endl;
				break;
			case 2:
				cout << "b!" << endl;
				break;
			case 3:
				cout << "c!" << endl;
				break;
			case 4:
				cout << "d!" << endl;
				break;
			case 5:
				cout << "e!" << endl;
				break;
			default:
				cout << "Hai inserito un numero non ammesso, inserisci di nuovo!" << endl;
				continue; //passo alla successiva iterazione del for: evito il break di uscita
		}
		break; //sono uscito correttamente dallo switch: interrompo il for infinito!
	}
	return 0;
}
	\end{lstlisting}
	In questo caso, non sappiamo quante volte l'utente sbaglierà ad inserire il numero: finché questo non è corretto dovremo continuare a chiederglielo. Un ciclo infinito si presta molto bene a questa situazione! Nel caso tutto vada bene, alla fine dello \textbf{switch}, vi è un \textbf{break} che interrompe il ciclo infinito. Se, invece, le cose non vanno come dovrebbero, nel \textbf{default} vi è un \textbf{continue} che fa saltare il \textbf{break}: si passa all'iterazione successiva!
	
	
	Avremmo potuto usare, al posto di un ciclo infinito, un \textbf{while} con, come condizione, un controllo sul numero inserito: finché rimane diverso da quelli ammessi, continuano le iterazioni. Questi sono modi diversi per scrivere lo stesso programma, ma l'ho detto: a me piace il ciclo \textbf{for}\ldots

